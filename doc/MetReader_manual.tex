\documentclass[11pt]{article}   % list options between brackets
%\usepackage{}              % list packages between braces
\usepackage{amsmath}
\usepackage{mathtools}
\usepackage{graphicx}
\usepackage{verbatim}

\setlength{\textwidth}{15.5cm}
\setlength{\oddsidemargin}{1.0cm}
\setlength{\evensidemargin}{1.0cm}
\setlength{\textheight}{21.0cm}
\setlength{\footskip}{1.0cm}
%\setlength\parindent{0pt}

\begin{document}

\title{MetReader: Documentation and User's Guide}   % type title between braces
\author{Hans F. Schwaiger}
\date{June 1, 2018}    % type date between braces
\maketitle

%%%%%%%%%%%%%%%%%%%%%%%%%%%%%%%%%%%%%%%%%%%%%%%%%%%%%%%%%%%%%%%%%%%%%%%%%%%%%%%
\section{Introduction}
MetReader is a library written in fortran 90 that provides an interface to
numerical weather prediction (NWP) data, or other forms of meteorological
data, such as radiosonde or other 1-d data.

This library was originally written as a component of the USGS volcanic ash
transport and dispersion model, Ash3d.  However, since it is useful for
other programs other than Ash3d, this interface to NWP files is provided
as a separate repository that can either be compiled as a library or simply
compiled directly with other source code.

NWP data are generally made available by agencies (NCEP, NOAA, NASA, etc.)
in a variety of formats (netcdf, grib1, grib2, hdf, ascii); each product having
its own data structure, naming convention, units, etc.  This library
isolates the calling program from the peculiarities of interfacing with
a particular NWP product.  Data can be returned to the calling program on
the native grid of the NWP product, or on any grid needed by the calling
program.  Projection and interpolation of NWP data to the required grid, along
with any rotation of velocity vectors to grid-relative, is calculated internally by
MetReader.
%\clearpage

%%%%%%%%%%%%%%%%%%%%%%%%%%%%%%%%%%%%%%%%%%%%%%%%%%%%%%%%%%%%%%%%%%%%%%%%%%%%%%%
\section{Installation}
This library requires two additional libraries made available on GitHub and USGS GitLab:
\begin{itemize}
\item HoursSince \\
This is a library that takes a date and time, and calculates the number of hours
since January 1, 1900 (or any artibrary base year).\\
(https://github.com/hschwaiger-usgs/HoursSince)
\item projection \\
This is a library that calculates commonly-used projected coordinates.\\
(https://github.com/hschwaiger-usgs/projection)
\end{itemize}

These libraries are installed by default into \texttt{/opt/USGS}.  If there are
installed elsewhere, then you will need to edit the makefile to point to your
installation.
Additionally, the default makefile will build MetReader with both netcdf and grib2
enabled.  If either of these libraries are unavailable on your system, you can
deactivate those options by setting the corresponding flags to \texttt{F} in the makefile.
On a RHEL/CentOS system, you can install the data format libraries by typing
\begin{verbatim}
     sudo yum install netcdf netcdf-devel grib_api grib_api-devel
\end{verbatim}
or for Ubuntu
\begin{verbatim}
     sudo apt-get install libnetcdf* libgrib-api*
\end{verbatim}
To compile MetReader as a library, simple type:
\begin{verbatim}
     make all
\end{verbatim}
This will build the requested components of the library.  If grib2 is enabled,
this will also build the grib2 indexer, make \texttt{gen\_GRIB2\_index}.
This is a tool that generates an index file of the grib records which speeds
access time to individual records substantially.

To install the library, module files and tools, edit the \texttt{INSTALLDIR} variable of
the makefile (the default is \texttt{/opt/USGS}) and type:
\begin{verbatim}
     make install
\end{verbatim}
This will also install scripts that can be used to download the 0.5 degree GFS
forecast files and the NCEP 2.5-degree Reannalysis files.

You will need to have write permission in \texttt{\$\{INSTALLDIR\}}
or install as root.  Assuming the libraries \texttt{projection} and \texttt{hourssince} are
installed in the default location, installation will result in the following
directory structure:
\begin{verbatim}
/opt/USGS
|-- bin
|   |-- autorun_scripts
|   |   |-- autorun_gfs0.5deg.sh
|   |   |-- autorun_NCEP_50YearReanalysis.sh
|   |   |-- convert_gfs0.5deg.sh
|   |   |-- get_gfs0.5deg.sh
|   |   |-- get_NCEP_50YearReanalysis.sh
|   |   `-- grib2nc.sh
|   |-- gen_GRIB2_index
|   |-- HoursSince1900
|   |-- ncMet_check
|   |-- ncMetSonde
|   |-- ncMetTraj_B
|   |-- ncMetTraj_F
|   `-- yyyymmddhh_since_1900
|-- include
|   |-- metreader.mod
|   `-- projection.mod
`-- lib
    |-- libhourssince.a
    |-- libMetReader.a
    `-- libprojection.a
\end{verbatim}

\clearpage
%%%%%%%%%%%%%%%%%%%%%%%%%%%%%%%%%%%%%%%%%%%%%%%%%%%%%%%%%%%%%%%%%%%%%%%%%%%%%%%
\section{Usage}
MetReader can read a variety of input data formats (1-d, 3-d, ASCII, netcdf, grib2) and
provide access to the meteorological data on either the native grid of the NWP data, or interpolated
onto a finer regular grid.  Before the meterological data can be accessible, several preliminary steps
must first be completed.
%%%%%%%%%%%%%%%%%%%%%%%%%%%%%%%%%%%%%%%%%%%%%%%%%%%%%%%%%%%
\subsection{Preliminary meta-data}
The calling program needs to prepare several parameters that define the type of meteorological
data to be read.  
\begin{itemize}
\item \texttt{iw} :: windfile class (this will be redundant/obselete soon)
 \begin{itemize}
 \item 1 : 1-D wind sounding
 \item 2 : 3-D grid is read from a ASCII file
 \item 3 : single, multistep 3-D datafile
 \item 4 : multiple 3-D datafiles, one per timestep
 \item 5 : special case for NCEP 50-year reanalysis and NOAA reanalysis
 \end{itemize}
\item \texttt{iwf} :: windfile format number (linked to specific products)
 \begin{itemize}
 \item   0 : Custom format based on template
 \item   1 : ASCII profile
 \item   2 : Radiosonde data
 \item   3 : North American Regional Reanalysis NARR-A (NCEP Grid 221; 32 km)
 \item   4 : North American Regional Reanalysis NARR-A+B (NCEP Grid 221; 32 km)
 \item   5 : NAM Regional Alaska (NCEP AWIPS Grid 216; 45 km)
 \item   6 : NAM N. Hemisphere (NCEP AWIPS Grid 104; 90 km)
 \item   7 : NAM Regional CONUS (NCEP AWIPS Grid 212; 40 km)
 \item   8 : NAM Regional CONUS (NCEP AWIPS Grid 218; 12km)
 \item   9 : Unassigned
 \item  10 : NAM Regional Alaska (NCEP AWIPS Grid 242; 11.25 km)
 \item  11 : NAM Regional Hawaii (NCEP Grid 196; 2.5 km)
 \item  12 : NAM Regional Alaska (NCEP Grid 198; 5.953 km)
 \item  13 : NAM Regional Alaska (NCEP Grid 91; 2.976 km)
 \item  20 : GFS (NCEP Grid 4; 0.5-degree)
 \item  21 : Unassigned
 \item  22 : GFS (NCEP Grid 193; 0.25-degree)
 \item  23 : NCEP-DOE Reanalysis 2 (NCEP Grid 2; 2.5-degree)
 \item  24 : NASA-MERRA Reanalysis (No NCEP Grid ID; 1.25-degree)
 \item  25 : NCEP/NCAR Reanalysis 1 (NCEP Grid 2; 2.5-degree)
 \item  27 : NOAA-CIRES 20th Century Reanalysis (NCEP Grid 2; 2.5-degree)
 \item  26 : Unassigned
 \item  28 : ECMWF ERA-Interim Reanalysis (NCEP Grid 170; Gaussian grid)
 \item  31 : Unassigned
 \item  32 : Air Force Weather Agency subcenter = 0 (No NCEP Grid ID; 0.25-degree)
 \item  33 : CCSM3.0 Community Atmosphere Model (No NCEP Grid ID; 3.75$\times$3.7-degree)
 \item  40 : NASA GEOS-5 Cp (No NCEP Grid ID; 0.625$\times$0.5-degree)
 \item  41 : NASA GEOS-5 Np (No NCEP Grid ID; 0.3125$\times$0.25-degree)
 \item  50 : WRF - output
 \item  51 : Unassigned
 \end{itemize}
\item \texttt{igrid} :: NCEP Grid ID
\item \texttt{idf} :: Data file format code (ascii, netcdf, grib, etc)
 \begin{itemize}
 \item   1 : ASCII
 \item   2 : NetCDF
 \item   3 : GRIB Edition 2
 \item   4 : GRIB Edition 1 (Not implemented; used by NARR)
 \item   5 : HDF (Not implemented; used by NASA MERRA)
 \end{itemize}
\item \texttt{iwfiles} :: number of windfiles to be read
%\item iy      ::   year of start time needed (integer)
\item \texttt{SimStartHour} ::  The starting time needed given as hours since 1900-01-01 00:00Z\\
This can alternatively be given as hours since a different base year as described below.
\item \texttt{SimDuration} ::  duration needed in hours
\end{itemize}

Additionally, there are several other parameters that could be set, if needed.
\begin{itemize}
\item \texttt{MR\_BaseYear} :: The default value for this is 1900, but must be
reassigned by the calling program is the wind files to be read contain data prior to 1900.
\item \texttt{MR\_useLeap} :: The default is \texttt{.true.} but can be set to \texttt{.false.}
if the wind files use a calendar without leap years (e.g. some paleoclimate CAM files)
\item \texttt{MR\_iHeightHandler} :: This code defines how MetReader behaves when values at
altitudes above those available in the wind files are requested.  Default value is 2.
 \begin{itemize}
 \item   1 : Stop the program
 \item   2 : Return wind values equal to the value at the highest available node; temperature
values remain constant from 11-20 km, then increase by $2^{\circ} \, \mathrm{C/km}$.
 \end{itemize}
\item \texttt{MR\_iwf\_template} :: This is the file name of the template for the
custom NetCDF files described in section \ref{SecCust}.  This is only required to be set
if \texttt{iwf}=0.  Currently, the custom windfile specification is only implemented for
\texttt{idf}=2 (NetCDF).
\item \texttt{call MR\_Reset\_Memory} :: If MetReader had been initialized and used for
one set of NWP files, but later during the execution of the calling program, a subsequent
set of NWP files is needed, this subroutine can be called to free all allocated memory.
\end{itemize}

\paragraph{Step 1:} The space needed for storing the wind file meta-data must be allocated.
\begin{verbatim}
     call MR_Allocate_FullMetFileList(iw,iwf,igrid,idf,iwfiles)
\end{verbatim}
This allocates the variable \texttt{MR\_windfiles(1:MR\_iwindfiles)}.

\paragraph{Step 2:} The calling program must now fill the list of file names \texttt{MR\_windfiles()}
with each name not exceeding 130 characters in length.  For the special cases for the NCEP/NCAR
Reanalysis 1 or the NOAA-CIRES 20th Century Reanalysis (\texttt{iwf=}25 or 27), MetReader expects
a specific structure to the directory holding the wind files.  For these reanalysis product,
the list of files is only one string (\texttt{iwfiles}=1)
with the `file' given as the root directory to the wind files.  For example, for the
NCEP/NCAR Reanalysis 1:
\begin{verbatim}
iwfiles = 1
MR_windfiles(1) = `/data/WindFiles/NCEP'
\end{verbatim}
where the files are stored as follows:
\begin{verbatim}
/data/WindFiles/NCEP
|-- 2016
|   |-- air.2016.nc
|   |-- hgt.2016.nc
|   |-- omega.2016.nc
|   |-- shum.2016.nc
|   |-- uwnd.2016.nc
|   `-- vwnd.2016.nc
|-- 2017
    |-- air.2017.nc
    |-- hgt.2017.nc
    |-- omega.2017.nc
    |-- shum.2017.nc
    |-- uwnd.2017.nc
    `-- vwnd.2017.nc
\end{verbatim}
Similarly, for the NOAA-CIRES 20th Century Reanalysis product:
\begin{verbatim}
iwfiles = 1
MR_windfiles(1) = `/data/WindFiles/NOAA'
\end{verbatim}
where the files are stored as follows:
\begin{verbatim}
/data/WindFiles/NOAA
|-- 2008
|   |-- GRIB
|   |   |-- pgrbanl_mean_2008_HGT_pres.grib
|   |   |-- pgrbanl_mean_2008_RH_pres.grib
|   |   |-- pgrbanl_mean_2008_TMP_pres.grib
|   |   |-- pgrbanl_mean_2008_UGRD_pres.grib
|   |   |-- pgrbanl_mean_2008_VGRD_pres.grib
|   |   `-- pgrbanl_mean_2008_VVEL_pres.grib
|   |-- pgrbanl_mean_2008_HGT_pres.nc
|   |-- pgrbanl_mean_2008_RH_pres.nc
|   |-- pgrbanl_mean_2008_TMP_pres.nc
|   |-- pgrbanl_mean_2008_UGRD_pres.nc
|   |-- pgrbanl_mean_2008_VGRD_pres.nc
|   `-- pgrbanl_mean_2008_VVEL_pres.nc
|-- 2009
    | -- ...
\end{verbatim}
Note that the grib files are in version 1 format and cannot yet be directly read by
MetReader.  They must first be converted to NetCDF and placed as described above.

\paragraph{Step 3:} Once the names of the wind files are specified, the files must
be queried to determine the scope of the available data.
\begin{verbatim}
     call MR_Read_Met_DimVars([iyear])
\end{verbatim}
The \texttt{iyear} argument is optional, but is expected for \texttt{iwf}=25 or 27
since MetReader needs to know whether to allocate space for 365 or 366 days.
Once the subroutine is called, each of the windfiles is checked for existence, then
the following public variables are set:
\begin{itemize}
\item All the projection parameters for the NWP grid.
\item \texttt{Met\_dim\_names} :: The names of the dimensions in the file in the order of: time,
pressure1, y, x, pressure2 (for Vz), pressure3 (for RH)
\item \texttt{Met\_var\_names} :: The names of the variables in the file according the the
following list:
\\
 Mechanical / State variables
 \begin{itemize}
  \item 1 = Geopotential Height ($\mathrm{gmp}$ or $\mathrm{m^2/s^2}$)
  \item 2 = Vx ($\mathrm{m/s}$)
  \item 3 = Vy ($\mathrm{m/s}$)
  \item 4 = Vz ($\mathrm{Pa/s}$)
  \item 5 = Temperature ($\mathrm{K}$)
 \end{itemize}
 Surface
 \begin{itemize}
  \item 10 = Planetary Boundary Layer Height ($\mathrm{m}$)
  \item 11 = U @ 10m ($\mathrm{m/s}$)
  \item 12 = V @ 10m ($\mathrm{m/s}$)
  \item 13 = Friction velocity ($\mathrm{m/s}$)
  \item 14 = Displacement Height ($\mathrm{m}$)
  \item 15 = Snow cover (\%)
  \item 16 = Soil moisture ($\mathrm{kg/m2}$)
  \item 17 = Surface Roughness ($\mathrm{m}$)
  \item 18 = Wind gust speed ($\mathrm{m/s}$)
  \item 19 = surface temperature ($\mathrm{K}$)
 \end{itemize}
 Atmospheric Structure
 \begin{itemize}
  \item 20 = pressure at lower cloud base ($\mathrm{Pa}$)
  \item 21 = pressure at lower cloud top ($\mathrm{Pa}$)
  \item 22 = temperature at lower cloud top ($\mathrm{K}$)
  \item 23 = Total Cloud cover (\%)
  \item 24 = Cloud cover (low) (\%)
  \item 25 = Cloud cover (convective) (\%)
 \end{itemize}
 Moisture
 \begin{itemize}
  \item 30 = Rel. Hum (\%)
  \item 31 = QV (specific humidity) ($\mathrm{kg/kg}$)
  \item 32 = QL (liquid) ($\mathrm{kg/kg}$)
  \item 33 = QI (ice) ($\mathrm{kg/kg}$)
 \end{itemize}
 Precipitation
 \begin{itemize}
  \item 40 = Categorical rain (0 or 1)
  \item 41 = Categorical snow (0 or 1)
  \item 42 = Categorical frozen rain (0 or 1)
  \item 43 = Categorical ice (0 or 1)
  \item 44 = Precipitation rate large-scale (liquid) ($\mathrm{kg/m^2s}$)
  \item 45 = Precipitation rate convective  (liquid) ($\mathrm{kg/m^2s}$)
  \item 46 = Precipitation rate large-scale (ice) ($\mathrm{kg/m^2s}$)
  \item 47 = Precipitation rate convective  (ice) ($\mathrm{kg/m^2s}$)
 \end{itemize}
\item \texttt{Met\_var\_conversion\_factor} :: For each of the variables, if the units
is not as listed above, then the correpsonding factor is set that converts to the
expected units.  For example, if precipitation rate is given in $\mathrm{mm/hr}$, then
the conversion factor to the expected units ($\mathrm{kg/m^2s}$) is $1/3600$ or
$2.778 \times 10^{-4}$.
\item \texttt{Met\_var\_IsAvailable} :: set to \texttt{.true.} if the NWP file
provides this variable.
\item \texttt{nx\_fullmet,ny\_fullmet,np\_fullmet} :: lengths of grid for x (lon), y (lat) and p
\item \texttt{x\_fullmet\_sp,y\_fullmet\_sp,p\_fullmet\_sp} :: values of the grid in km (degrees) and hPa
\item \texttt{IsLatLon\_MetGrid} :: \texttt{.true.} if the grid is specified in longitude and latitude
\item \texttt{IsGlobal\_MetGrid} :: \texttt{.true.} if the grid is periodic in longitude.  (Note:
mapping across the poles is not yet implemented).
\item \texttt{IsRegular\_MetGrid} :: \texttt{.true.} if $\mathrm{dx,dy}$ ($\mathrm{dlon,dlat}$) is constant
\end{itemize}

\paragraph{Step 4:} Next, the calling program needs to define the type of grid onto which MetReader 
will interpolate values.  Namely, whether or not the computational grid is projectected, and if so,
which projection and associated values.  This grid used by the calling program can be independent
of the grid used by the NWP files.  The parameter list for the subroutine below are the values
needed by the \texttt{libprojection.a} library.  Below is an example call from Ash3d.
\begin{verbatim}
     call MR_Set_CompProjection(IsLatLon,A3d_iprojflag,A3d_lam0,           &
                                         A3d_phi0,A3d_phi1,A3d_phi2,       &
                                         A3d_k0_scale,A3d_radius_earth)
\end{verbatim}
If \texttt{IsLatLon=.true.} then all other projection parameters are ignored.  Similarly,
some of the parameters are required and others not, depending on the projection used.
Currently, \texttt{libprojection.a} has implemented
Polar stereographic (\texttt{iprojflag=1}),
Albers Equal Area (\texttt{iprojflag=2}),
UTM (\texttt{iprojflag=3}),
Lambert conformal conic (\texttt{iprojflag=4}),
and Mercator (\texttt{iprojflag=5}).

\paragraph{Step 5:} The final prepatory step is the evaluate the windfiles provided
with the spatial and temporal requirements of the calling program.
\begin{verbatim}
     call MR_Initialize_Met_Grids(nxmax,nymax,nzmax,                &
                                  x(1:nxmax),y(1:nymax),z(1:nzmax), &
                                  IsPeriodic)
\end{verbatim}
\texttt{nxmax,nymax,nzmax} are the sizes of the computational grid. \texttt{x,y,z} are the
single-precision grid values in km (or degrees).  \texttt{IsPeriodic} indicates whether or
not the computational grid is periodic in x.  This subroutine also determines the size
of the sub-grid of the full NWP data that is needed by the calling program.
\begin{verbatim}
     call MR_Set_Met_Times(SimStartHour, SimDuration)
\end{verbatim}
Here \texttt{SimStartHour} and \texttt{SimDuration} give the initial time (in hours since 1900 (or
other base year)) and the length of time needed in hours.  This subroutine opens all the
files listed in \texttt{MR\_windfiles()} and verifies that the files provided cover the requested
time span.  Then a list is prepared of all the time steps and files needed to fully cover the 
requested time with the step index \texttt{istep=1} the step at or just prior to \texttt{SimStartHour}.
The hours between time steps is saved in \texttt{MR\_MetStep\_Interval()} and does not need to
be constant.

%%%%%%%%%%%%%%%%%%%%%%%%%%%%%%%%%%%%%%%%%%%%%%%%%%%%%%%%%%%
\subsection{Reading meteorological data}
When handling data from the NWP files, there are three grids that MetReader uses:
\begin{itemize}
\item \texttt{metP} :: the native grid of the NWP file clipped to span the
computational needs (possible re-ordered)
\item \texttt{metH} :: the horizontal grid of the NWP file, but with the vertical
coordinate mapped from the pressure grid of the NWP file to the altitude grid of 
the computational grid.
\item \texttt{compH} :: the full x,y,z of the computational grid.
\end{itemize}
Once the computational grid is specified and the sub-grid of the NWP is determined in 
step 5 above, the following public work spaces are allocated.
\begin{verbatim}
     MR_dum2d_met_int(nx_submet,ny_submet)          :: integer
     MR_dum2d_met(nx_submet,ny_submet)              :: float
     MR_dum3d_metP(nx_submet,ny_submet,np_fullmet)  :: float
     MR_dum3d2_metP(nx_submet,ny_submet,np_fullmet) :: float
     MR_dum3d_metH(nx_submet,ny_submet,nz_comp)     :: float
     MR_dum2d_comp_int(nx_comp,ny_comp)             :: integer
     MR_dum2d_comp(nx_comp,ny_comp)                 :: float
     MR_dum3d_compH(nx_comp,ny_comp,nz_comp)        :: float
\end{verbatim}
\texttt{nx\_comp}, \texttt{ny\_comp}, \texttt{np\_comp} are the number
of nodes in the computational grid and 
\texttt{nx\_submet}, \texttt{ny\_submet}, \texttt{np\_submet} are the
number of nodes of the NWP grid that is needed
to span the required computational grid.

To read data from the meteorological files, the fundamental subroutine is
\begin{verbatim}
     call MR_Read_3d_MetP_Variable(ivar,istep)
\end{verbatim}
where \texttt{ivar} is the code for the variable of interests (see table %\ref{TabVar}
) and
\texttt{istep} is the index of the sequence of time steps set up based on
\texttt{SimStartHour} and \texttt{SimDuration} in Step 5 above.  This subroutine provides
a common interface regardless of the data format (ASCII, NetCDF, GRIB2, etc.), data
structure (latitude might be top-to-bottom or bottom-to-top, pressure might be surface-to-top
or top-down, grid might be staggered, longitude might be $-180 \rightarrow 180$ or $0 \rightarrow 360$).
In all cases for 3-d variables, what is returned is the public work array
\texttt{MR\_dum3d\_metP}
This array is consistantly ordered such that \texttt{x}
and \texttt{y} (or \texttt{lon} and \texttt{lat}) increase with index (North to South NWP grids
are flipped) and with \texttt{p} ordered from the surface to top (pressure is high to low).

If 3-d data is needed on the \texttt{metH} grid,
\begin{verbatim}
     call MR_Read_3d_MetH_Variable(ivar,istep)
\end{verbatim}
can be used which first populates \texttt{MR\_dum3d\_metP}, then interpolates
onto \texttt{MR\_dum3d\_metH}.

For simply reading meteorological data onto the computational grid, use
\begin{verbatim}
     call MR_Read_3d_Met_Variable_to_CompGrid(ivar,istep,[IsNext])
\end{verbatim}
This subroutine reads the variable \texttt{ivar} from the NWP file into
\texttt{MR\_dum3d\_metP}, then interpolates these values onto \texttt{MR\_dum3d\_metH},
finally interpolating onto \texttt{MR\_dum3d\_compH}.  The optional argument
\texttt{IsNext} is used when velocities are used at different points in the program
and need to be saved.  \texttt{IsNext} indicates that the velocity values
\texttt{MR\_v[x,y]\_metP\_next} should be copied to \texttt{MR\_v[x,y]\_metP\_last.}

To interpolate 3-d data onto the altitude levels of the computational grid, first
the geopotential height data must be read.  This could be achieved by the subroutine
\texttt{MR\_Read\_3d\_MetP\_Variable} described above, but as a simulation advances, we
typically need to have the geopotential height data for the previous and the next time
steps so that values can be interpolated between steps.  The subroutine
\begin{verbatim}
     call MR_Read_HGT_arrays(istep,[reset_first_time])
\end{verbatim}
is a specialized reader for \texttt{ivar=1}.  The first time this is called, both
\texttt{istep} and \texttt{istep+1} are called, populating the two public variables
\texttt{MR\_geoH\_metP\_last} and \texttt{MR\_geoH\_metP\_next}.  Subsequent calls
copy \texttt{MR\_geoH\_metP\_next} to \texttt{MR\_geoH\_metP\_last} and reads
the next values from \texttt{istep+1}.  If the optional argument \texttt{reset\_first\_time}
is given, both \texttt{istep} and \texttt{istep+1} are directly read.


Similar read subroutines are available for 2d variables.
\begin{verbatim}
     call MR_Read_2d_Met_Variable(ivar,istep)
\end{verbatim}
Given a 2-d variable identified with \texttt{ivar}, this populates the public
work array \texttt{MR\_dum2d\_met}, where again,
\texttt{1:nx\_submet,1:ny\_submet} corresponds to the subset of nodes of the NWP
file that is needed to span the computational grid.
\begin{verbatim}
     call MR_Read_2d_Met_Variable_to_CompGrid(ivar,istep)
\end{verbatim}
can be called similar to \texttt{MR\_Read\_2d\_Met\_Variable} which directly reads the
variable needed and returns the
values interpolated onto the public work space \texttt{MR\_dum2d\_comp}.

\subsubsection{Regridding meteorological data}
Internal to MetReader are a variety of regridding subroutines to interpolate from
the \texttt{metP} grid onto \texttt{metH} or \texttt{compH}.  
In some circumstances, data is initially needed just on the native NWP grid, but
then derived values might be required on the full computational grid.
For example, if air viscosity is needed for a particle fall model, temperature and
pressure values on the NWP grid can be read.  Then viscosities can be calculated for each
of the corresponding nodes of the NWP grid.  To interpolate these values from the 
\texttt{metP} grid to the \texttt{compH} grid, the subroutine
\begin{verbatim}
     call MR_Regrid_MetP_to_CompGrid(istep)
\end{verbatim}
can be used.  This takes the variable \texttt{MR\_dum3d\_metP} and interpolates it
onto \texttt{MR\_dum3d\_compH}.  Similarly, 
\begin{verbatim}
     call MR_Regrid_MetP_to_MetH(istep)
\end{verbatim}
interpolates \texttt{MR\_dum3d\_metP} to \texttt{MR\_dum3d\_metH} and 
\begin{verbatim}
     call MR_Regrid_Met2d_to_Comp2d
\end{verbatim}
interpolates \texttt{MR\_dum2d\_met} to \texttt{MR\_dum2d\_comp}.


\subsubsection{Wind velocity vectors}
In some cases, the wind velocity vectors need to be mapped from the coordinate
system of the NWP files to the computational grid.  Velocity components on the
computational grid are always provided as grid-relative.  For some NWP files, such
as the North American Regional Reanalysis (NARR), the NWP grid is projected, yet
the wind components are provided as Earth-relative.  In other cases, if the NWP
data are provided on a projected grid, but the computational grid is either lon/lat
or on a different projection, the grid-relative, projected velocity components must
be converted to that needed for the computational grid.

If the wind data are provided as grid-relative on a projected grid and something else
is needed, the subroutine
\begin{verbatim}
     call MR_Rotate_UV_GR2ER_Met(MR_iMetStep_Now)
\end{verbatim}
will read both U and V components and decompose the vector into earth-relative Easterly
and Northerly wind components with values at the \texttt{metP} grid nodes.  These
components are put in the variables \texttt{MR\_u\_ER\_metP} and \texttt{MR\_v\_ER\_metP}.
Once we have Earth-relative velocity components (either from direct read of lon/lat data,
from the subrouting \texttt{MR\_Rotate\_UV\_GR2ER\_Met} or fron direct read of NARR data),
if the computational grid is not Earth-relative, then these Earth-relative components can be
subsequently decomposed into grid-relative components on the computational grid with
\begin{verbatim}
     call MR_Rotate_UV_ER2GR_Comp(MR_iMetStep_Now)
\end{verbatim}
This subroutine returns the U and V components on the \texttt{compH} grid through the two
dummy work space variables \texttt{MR\_dum3d\_compH} and \texttt{MR\_dum3d\_compH\_2}

\subsection{Additional Funtions}
\paragraph{Horizontal derivatives}
MetReader has subroutines for calculating horizontal derivitives of variables on the
\texttt{metP} grid.  This is used for calculating velocity gradients for diffusivity
calculations, but could be applied to any variable.  The two subroutines are
\texttt{MR\_DelMetP\_Dx} and \texttt{MR\_DelMetP\_Dy}.  Both read the values from
\texttt{MR\_dum3d\_metP} and return derivative values on \texttt{MR\_dum3d2\_metP}.

\paragraph{U.S. Standard Atmosphere}
MetReader also has several functions for calculating values from the U.S. Standard
Atmosphere.
\begin{itemize}
\item  \texttt{MR\_Temp\_US\_StdAtm(zin)} :: Returns a temperature in $\mathrm{K}$
given a height in $\mathrm{km}$.
\item  \texttt{MR\_Z\_US\_StdAtm(pin)} :: Returns a height in $\mathrm{km}$ given a
pressure in $\mathrm{hPa}$.
\item  \texttt{MR\_Pres\_US\_StdAtm(zin)} :: Returns a pressure in $\mathrm{hPa}$ given
a height in $\mathrm{km}$.
\end{itemize}


%%%%%%%%%%%%%%%%%%%%%%%%%%%%%%%%%%%%%%%%%%%%%%%%%%%%%%%%%%%%%%%%%%%%%%%%%%%%%%%
\section{Supported types of meteorological data}
MetReader can read 
\subsection{1-d user-specified profiles}
\subsection{1-d radiosonde profiles (single or multiple times)}
\subsection{Network of 1-d radiosonde profiles}
\subsection{3-d forecast or reanalysis data}
\subsubsection{ASCII input}
\subsubsection{netcdf}
\subsubsection{hdf}
\subsubsection{grib2}

%\section{Usage: For programs with no computational grid}

%\clearpage
%\section{Usage: For programs with a separate computational grid}

%\clearpage
%\section{Library structure}
%MetReader.f90
%  subroutines:
%    MR_Allocate_MetFileList
%    MR_Set_CompProjection
%    MR_Initialize_Met_Grids
%    MR_Set_Met_Times
%    MR_Read_HGT_arrays
%    MR_Read_3d_MetP_Variable
%    MR_Read_3d_MetH_Variable
%    MR_Read_3d_Met_Variable_to_CompGrid
%    MR_Read_2d_Met_Variable
%    MR_Read_2d_Met_Variable_to_CompGrid
%    MR_Read_UV_MetGridRelative
%    MR_Read_UV_CompGridRelative
%    MR_Regrid_MetP_to_CompGrid
%    MR_Regrid_MetP_to_MetH
%    MR_Regrid_Met2d_to_Comp2d
%    MR_DelMetP_Dx
%    MR_DelMetP_Dy
%    MR_GridChecker
%  functions:
%    MR_Temp_US_StdAtm
%    MR_Z_US_StdAtm
%    MR_Pres_US_StdAtm
%
%MetReader_Grids.f90
%  subroutines:
%    MR_Set_Met_NCEPGeoGrid
%    MR_Set_Comp2Met_Map
%    MR_Regrid_Met2Comp
%  functions:
%
%MetReader_ASCII.f90
%  subroutines:
%    MR_Set_MetComp_Grids_1dascii
%    MR_Set_Met_Times_ascii
%  functions:
%
%MetReader_NetCDF.f90
%  subroutines:
%  functions:
%
%MetReader_HWM_NRLMSISE.f90
%  subroutines:
%    Allocate_HWT_supl_grid
%    Allocate_HWT_full_grid
%    Set_HWT_UVT
%    Get_WindTemp_Empir
%  functions:

\clearpage
\section{Supported meteorological files / grids}

\clearpage
\section{Custom meteorlogical file specification}\label{SecCust}
\begin{figure}[htbp]\begin{center}
 \includegraphics[angle=-90,scale=1.0]{Figs/Geostationary_NAM-Grids.pdf}
\parbox{15cm}{\caption{\label{FigNAMs}
North America Mesoscale models from NCEP
}}
\end{center}\end{figure}

%%%%%%%%%%%%%%%%%%%%%%%%%%%%%%%%%%%%%%%%%%%%%%%%%%%%
% Broader North America
%%%%%%%%%%%%%%%%%%%%%%%%%%%%%%%%%%%%%%
\clearpage
\subsection{NAM 5 Regional N.America (190.5 km)}

\verb|nam.t00z.grb5fm00.tm00|\\
Available from:\\
\verb|ftp://ftp.ncep.noaa.gov/pub/data/nccf/com/nam/prod/nam.YYYYMDD/| \\

\begin{figure}[htbp]\begin{center}
% \includegraphics[angle=-90,scale=0.9]{Figs/AK_n005.pdf}
\parbox{15cm}{\caption{\label{FigNAM005}
NAM 5, Regional N. America 190.5 km Polar Stereographic
}}
\end{center}\end{figure}
\clearpage
\verb|n005_template.txt| \\
\tiny \verbatiminput{n005_template.txt} \normalsize


%%%%%%%%%%%%%%%%%%%%%%%%%%%%%%%%%%%%%%
\clearpage
\subsection{NAM 104 Regional N.America (90.7 km)}

\verb|nam.t00z.grbgrd00.tm00|\\
Available from:\\
\verb|ftp://ftp.ncep.noaa.gov/pub/data/nccf/com/nam/prod/nam.YYYYMDD/| \\

\begin{figure}[htbp]\begin{center}
% \includegraphics[angle=-90,scale=0.5]{Figs/grid104.pdf}
\parbox{15cm}{\caption{\label{FigNAM005}
NAM 104, Regional N. America 90.7 km Polar Stereographic
}}
\end{center}\end{figure}
\clearpage
\verb|n104_template.txt| \\
\tiny \verbatiminput{n104_template.txt} \normalsize



%%%%%%%%%%%%%%%%%%%%%%%%%%%%%%%%%%%%%%
\clearpage
\subsection{NAM 221 Regional N.America 32.5 km)}

\verb|nam.t00z.awip3200.tm00|\\
Available from:\\
\verb|ftp://ftp.ncep.noaa.gov/pub/data/nccf/com/nam/prod/nam.YYYYMDD/| \\

\begin{figure}[htbp]\begin{center}
% \includegraphics[angle=-90,scale=1.0]{Figs/AK_n221.pdf}
\parbox{15cm}{\caption{\label{FigNAM221}
NAM 221, Regional N. America 32.5 km Lambert Conformal Conical
}}
\end{center}\end{figure}
\clearpage
\verb|n221_template.txt| \\
\tiny \verbatiminput{n221_template.txt} \normalsize


%%%%%%%%%%%%%%%%%%%%%%%%%%%%%%%%%%%%%%
% Alaska
%%%%%%%%%%%%%%%%%%%%%%%%%%%%%%%%%%%%%%
\clearpage
\subsection{NAM 216 AK (45.0 km)}

\verb|nam.t00z.awipak00.tm00|\\
Available from:\\
\verb|ftp://ftp.ncep.noaa.gov/pub/data/nccf/com/nam/prod/nam.YYYYMDD/| \\
\verb|http://motherlode.ucar.edu/native/conduit/data/nccf/com/nam/prod/nam.YYYYMMDD/|\\

\begin{figure}[htbp]\begin{center}
 %\includegraphics[angle=-90,scale=1.0]{Figs/AK_n216.pdf}
\parbox{15cm}{\caption{\label{FigNAM2116}
NAM 216, AK 45.0 km Polar Stereographic
}}
\end{center}\end{figure}
\clearpage 
\verb|n216_template.txt| \\
\tiny \verbatiminput{n216_template.txt} \normalsize


%%%%%%%%%%%%%%%%%%%%%%%%%%%%%%%%%%%%%%
\clearpage
\subsection{NAM 242 AK (11.25km)}

\verb|nam.t00z.awak3d00.grb2.tm00|\\
Available from:\\
\verb|ftp://ftp.ncep.noaa.gov/pub/data/nccf/com/nam/prod/nam.YYYYMDD/|

\begin{figure}[htbp]\begin{center}
% \includegraphics[angle=-90,scale=1.0]{Figs/AK_n242.pdf}
\parbox{15cm}{\caption{\label{FigNAM242}
NAM 242, AK 11.25 km Polar Stereographic
}}
\end{center}\end{figure}
\clearpage 
\verb|n242_template.txt| \\
\tiny \verbatiminput{n242_template.txt} \normalsize

%%%%%%%%%%%%%%%%%%%%%%%%%%%%%%%%%%%%%%
\clearpage
\subsection{NAM 198 AK (5.9 km)}

\verb|nam.t00z.alaskanest.hiresf00.tm00|\\
Available from:\\
\verb|ftp://ftp.ncep.noaa.gov/pub/data/nccf/com/nam/prod/nam.YYYYMDD/| \\

\begin{figure}[htbp]\begin{center}
% \includegraphics[angle=-90,scale=1.0]{Figs/AK_n198.pdf}
\parbox{15cm}{\caption{\label{FigNAM198}
NAM 198, AK 5.9 km Polar Stereographic
}}
\end{center}\end{figure}
\clearpage
\verb|n198_template.txt| \\
\tiny \verbatiminput{n198_template.txt} \normalsize


%%%%%%%%%%%%%%%%%%%%%%%%%%%%%%%%%%%%%%
% North Pacific / Hawaii
%%%%%%%%%%%%%%%%%%%%%%%%%%%%%%%%%%%%%%
\clearpage
\subsection{NAM 243 Eastern N. Pac./HI (0.40 degrees)}

\verb|nam.t00z.awiphi00.tm00|\\
Available from:\\
\verb|ftp://ftp.ncep.noaa.gov/pub/data/nccf/com/nam/prod/nam.YYYYMDD/|

\begin{figure}[htbp]\begin{center}
% \includegraphics[angle=-90,scale=0.9]{Figs/AK_n243.pdf}
\parbox{15cm}{\caption{\label{FigNAM243}
NAM 243, Eastern N. Pac./HI 0.4-degrees
}}
\end{center}\end{figure}
\clearpage
\verb|n243_template.txt| \\
\tiny \verbatiminput{n243_template.txt} \normalsize

%%%%%%%%%%%%%%%%%%%%%%%%%%%%%%%%%%%%%%
\clearpage
\subsection{NAM 182 HI (0.108 degrees)}

\verb|nam.t00z.afwahi00.grb2.tm00|\\
Available from:\\
\verb|ftp://ftp.ncep.noaa.gov/pub/data/nccf/com/nam/prod/nam.YYYYMDD/|

\begin{figure}[htbp]\begin{center}
% \includegraphics[angle=-90,scale=0.9]{Figs/AK_n182.pdf}
\parbox{15cm}{\caption{\label{FigNAM182}
NAM 182, HI 0.108-degrees
}}
\end{center}\end{figure}
\clearpage
\verb|n182_template.txt| \\
\tiny \verbatiminput{n182_template.txt} \normalsize


%%%%%%%%%%%%%%%%%%%%%%%%%%%%%%%%%%%%%%
\clearpage
\subsection{NAM 196 HI (2.5km)}

\verb|nam.t00z.hawaiinest.hiresf00.tm0|\\
Available from:\\
\verb|ftp://ftp.ncep.noaa.gov/pub/data/nccf/com/nam/prod/nam.YYYYMDD/|

\begin{figure}[htbp]\begin{center}
% \includegraphics[angle=-90,scale=1.0]{Figs/AK_n196.pdf}
\parbox{15cm}{\caption{\label{FigNAM196}
NAM 196, AK 2.5 km Mercator
}}
\end{center}\end{figure}
\clearpage
\verb|n196_template.txt| \\
\tiny \verbatiminput{n196_template.txt} \normalsize

%%%%%%%%%%%%%%%%%%%%%%%%%%%%%%%%%%%%%%
% Continental U.S.
%%%%%%%%%%%%%%%%%%%%%%%%%%%%%%%%%%%%%%
\clearpage
\subsection{NAM 211 CONUS (81.3 km)}

\verb|nam.t00z.awp21100.tm00|\\
Available from:\\
\verb|ftp://ftp.ncep.noaa.gov/pub/data/nccf/com/nam/prod/nam.YYYYMDD/|

\begin{figure}[htbp]\begin{center}
% \includegraphics[angle=-90,scale=1.0]{Figs/AK_n211.pdf}
\parbox{15cm}{\caption{\label{FigNAM211}
NAM 211 CONUS Lambert Conformal Conical
}}
\end{center}\end{figure}
\clearpage
\verb|n211_template.txt| \\
\tiny \verbatiminput{n211_template.txt} \normalsize

%%%%%%%%%%%%%%%%%%%%%%%%%%%%%%%%%%%%%%
\clearpage
\subsection{NAM 212 CONUS (40.6 km)}

\verb|nam.t00z.awip3d00.tm00|\\
Available from:\\
\verb|ftp://ftp.ncep.noaa.gov/pub/data/nccf/com/nam/prod/nam.YYYYMDD/|

\begin{figure}[htbp]\begin{center}
% \includegraphics[angle=-90,scale=1.0]{Figs/AK_n212.pdf}
\parbox{15cm}{\caption{\label{FigNAM212}
NAM 212 CONUS Lambert Conformal Conical
}}
\end{center}\end{figure}
\clearpage
\verb|n212_template.txt| \\
\tiny \verbatiminput{n212_template.txt} \normalsize

%%%%%%%%%%%%%%%%%%%%%%%%%%%%%%%%%%%%%%
\clearpage
\subsection{NAM 218 CONUS (12.2 km)}

\verb|nam.t00z.awphys00.grb2.tm00|\\
Available from:\\
\verb|ftp://ftp.ncep.noaa.gov/pub/data/nccf/com/nam/prod/nam.YYYYMDD/|

\begin{figure}[htbp]\begin{center}
% \includegraphics[angle=-90,scale=1.0]{Figs/AK_n218.pdf}
\parbox{15cm}{\caption{\label{FigNAM218}
NAM 218 CONUS Lambert Conformal Conical
}}
\end{center}\end{figure}
\clearpage
\verb|n218_template.txt| \\
\tiny \verbatiminput{n218_template.txt} \normalsize

%%%%%%%%%%%%%%%%%%%%%%%%%%%%%%%%%%%%%%
\clearpage
\subsection{NAM 227 CONUS (5.1 km)}

\verb|nam.t00z.conusnest.hiresf00.tm00|\\
Available from:\\
\verb|ftp://ftp.ncep.noaa.gov/pub/data/nccf/com/nam/prod/nam.YYYYMDD/|

\begin{figure}[htbp]\begin{center}
% \includegraphics[angle=-90,scale=1.0]{Figs/AK_n227.pdf}
\parbox{15cm}{\caption{\label{FigNAM218}
NAM 227 CONUS Lambert Conformal Conical
}}
\end{center}\end{figure}
\clearpage
\verb|n227_template.txt| \\
\tiny \verbatiminput{n227_template.txt} \normalsize

%%%%%%%%%%%%%%%%%%%%%%%%%%%%%%%%%%%%%%
% Eastern U.S, / Caribbean
%%%%%%%%%%%%%%%%%%%%%%%%%%%%%%%%%%%%%%
\clearpage
\subsection{NAM 181 Caribbean (0.108 degrees)}

\verb|nam.t00z.afwaca00.grb2.tm00|\\
Available from:\\
\verb|ftp://ftp.ncep.noaa.gov/pub/data/nccf/com/nam/prod/nam.YYYYMDD/|

\begin{figure}[htbp]\begin{center}
% \includegraphics[angle=-90,scale=0.9]{Figs/AK_n181.pdf}
\parbox{15cm}{\caption{\label{FigNAM181}
NAM 181, Caribbean 0.108-degrees
}}
\end{center}\end{figure}
\clearpage
\verb|n181_template.txt| \\
\tiny \verbatiminput{n181_template.txt} \normalsize


%%%%%%%%%%%%%%%%%%%%%%%%%%%%%%%%%%%%%%
\clearpage
\subsection{NAM 194 Puerto Rico (2.5 km)}

\verb|nam.t00z.priconest.hiresf00.tm00|\\
Available from:\\
\verb|ftp://ftp.ncep.noaa.gov/pub/data/nccf/com/nam/prod/nam.YYYYMDD/|

\begin{figure}[htbp]\begin{center}
% \includegraphics[angle=-90,scale=0.9]{Figs/AK_n194.pdf}
\parbox{15cm}{\caption{\label{FigNAM194}
NAM 194, Puerto Rico Mercator 2.5 km
}}
\end{center}\end{figure}
\clearpage
\verb|n194_template.txt| \\
\tiny \verbatiminput{n194_template.txt} \normalsize
%
%

%%%%%%%%%%%%%%%%%%%%%%%%%%%%%%%%%%%%%%
% Global 
%%%%%%%%%%%%%%%%%%%%%%%%%%%%%%%%%%%%%%
\clearpage
\subsection{GFS 2 (2.50 degrees)}

\verb|gfs.t00z.pgrb2.2p50.f000.nc|\\
Available from:\\
\verb|ftp://ftp.ncep.noaa.gov/pub/data/nccf/com/gfs/prod/gfs.YYYYMDD/|

\begin{figure}[htbp]\begin{center}
% \includegraphics[angle=0,scale=0.9]{Figs/AK_n002.pdf}
\parbox{15cm}{\caption{\label{FigNAM002}
GFS 2, 2.5-degrees
}}
\end{center}\end{figure}
\clearpage
\verb|n002_template.txt| \\
\tiny \verbatiminput{n002_template.txt} \normalsize

%%%%%%%%%%%%%%%%%%%%%%%%%%%%%%%%%%%%%%
\clearpage
\subsection{GFS 3 (1.00 degrees)}

\verb|gfs.t00z.pgrb2.1p00.f000.nc|\\
Available from:\\
\verb|ftp://ftp.ncep.noaa.gov/pub/data/nccf/com/gfs/prod/gfs.YYYYMDD/|

\begin{figure}[htbp]\begin{center}
% \includegraphics[angle=0,scale=0.9]{Figs/AK_n003.pdf}
\parbox{15cm}{\caption{\label{FigNAM003}
GFS 3, 1.0-degrees
}}
\end{center}\end{figure}
\clearpage
\verb|n003_template.txt| \\
\tiny \verbatiminput{n003_template.txt} \normalsize

%%%%%%%%%%%%%%%%%%%%%%%%%%%%%%%%%%%%%%
\clearpage
\subsection{GFS 4 (0.50 degrees)}

\verb|gfs.t00z.pgrb2.0p50.f000.nc|\\
Available from:\\
\verb|ftp://ftp.ncep.noaa.gov/pub/data/nccf/com/gfs/prod/gfs.YYYYMDD/|

\begin{figure}[htbp]\begin{center}
% \includegraphics[angle=0,scale=0.9]{Figs/AK_n004.pdf}
\parbox{15cm}{\caption{\label{FigNAM004}
GFS 4, 0.5-degrees
}}
\end{center}\end{figure}
\clearpage
\verb|n004_template.txt| \\
\tiny \verbatiminput{n004_template.txt} \normalsize

%%%%%%%%%%%%%%%%%%%%%%%%%%%%%%%%%%%%%%
\clearpage
\subsection{GFS 193 (0.25 degrees)}

\verb|gfs.t00z.pgrb2.0p25.f000.nc|\\
Available from:\\
\verb|ftp://ftp.ncep.noaa.gov/pub/data/nccf/com/gfs/prod/gfs.YYYYMDD/|

\begin{figure}[htbp]\begin{center}
% \includegraphics[angle=0,scale=0.9]{Figs/AK_n193.pdf}
\parbox{15cm}{\caption{\label{FigNAM002}
GFS 193, 0.25-degrees
}}
\end{center}\end{figure}
\clearpage
\verb|n193_template.txt| \\
\tiny \verbatiminput{n193_template.txt} \normalsize


%%%%%%%%%%%%%%%%%%%%%%%%%%%%%%%%%%%%%%
\clearpage
\subsection{NASA GEOS-5 (0.625/0.5 degrees)}

\verb|GEOS.fp.fcst.inst3_3d_asm_Cp.YYYYMMDD_00+YYYYMMDD_mmmm.V01.nc4|\\
Available from:\\
\verb|ftp://gmao_ops@ftp.nccs.nasa.gov/fp/forecast//|

\begin{figure}[htbp]\begin{center}
% \includegraphics[angle=0,scale=0.9]{Figs/AK_nGCp.pdf}
\parbox{15cm}{\caption{\label{FigGCp}
NASA GEOS-5, 0.625/0.5-degrees
}}
\end{center}\end{figure}
\clearpage
\verb|nGCp_template.txt| \\
\tiny \verbatiminput{nGCp_template.txt} \normalsize

%%%%%%%%%%%%%%%%%%%%%%%%%%%%%%%%%%%%%%
\clearpage
\subsection{NASA GEOS-5 (0.25/0.3125 degrees)}

\verb|GEOS.fp.fcst.inst3_3d_asm_Np.YYYYMMDD_00+YYYYMMDD_mmmm.V01.nc4|\\
Available from:\\
\verb|ftp://gmao_ops@ftp.nccs.nasa.gov/fp/forecast//|

\begin{figure}[htbp]\begin{center}
% \includegraphics[angle=0,scale=0.9]{Figs/AK_nGNp.pdf}
\parbox{15cm}{\caption{\label{FigGNp}
NASA GEOS-5, 0.25/0.3125-degrees
}}
\end{center}\end{figure}
\clearpage
\verb|nGNp_template.txt| \\
\tiny \verbatiminput{nGNp_template.txt} \normalsize

%%%%%%%%%%%%%%%%%%%%%%%%%%%%%%%%%%%%%%
\clearpage
\subsection{NASA MERRA 1.25 degrees)}

\verb|MERRA300.prod.assim.inst3_3d_asm_Cp.YYYYMMDD.hdf|\\
Available from:\\
\verb|ftp://goldsmr3.sci.gsfc.nasa.gov/data/s4pa/MERRA/MAI3CPASM.5.2.0/|

\begin{figure}[htbp]\begin{center}
% \includegraphics[angle=0,scale=0.9]{Figs/AK_nMCp.pdf}
\parbox{15cm}{\caption{\label{FigMCp}
NASA MERRA 1.25-degrees
}}
\end{center}\end{figure}
\clearpage
\verb|nMCp_template.txt| \\
\tiny \verbatiminput{nMCp_template.txt} \normalsize


%
%NCEP
%  REFTIME  => time(attribute='units')="Hour since 2016-01-25T00:00:00Z"
%  STEPTIME => time(i) = double hours
%
%	double time(time) ;
%		time:units = "Hour since 2016-01-25T00:00:00Z" ;
%		time:standard_name = "time" ;
%		time:long_name = "GRIB forecast or observation time" ;
%Catania
%  REFTIME  => reftime="2015 03 25 00:0"
%  STEPTIME => frtime(i) = float hours
%
%	float frtime(frtime) ;
%		frtime:units = "hours" ;
%	char reftime(timelen) ;
%             reftime = "2015 03 25 00:00" ;
%
%
%
%NASA
%  REFTIME  => time(attribute='units')="minutes since 2016-01-25 03:00:00"
%  STEPTIME => time(i) = integer minutes
%
%	int time(time) ;
%		time:long_name = "time" ;
%		time:units = "minutes since 2016-01-25 03:00:00" ;
%		time:time_increment = 30000 ;
%		time:begin_date = 20160125 ;
%		time:begin_time = 30000 ;
%		time:vmax = 1.e+15f ;
%		time:vmin = -1.e+15f ;
%		time:valid_range = -1.e+15f, 1.e+15f ;
%        double TIME_EOSGRID(TIME_EOSGRID) ;
%                TIME_EOSGRID:hdf_name = "TIME:EOSGRID" ;
%                TIME_EOSGRID:begin_time = 0 ;
%                TIME_EOSGRID:begin_date = 20150721 ;
%                TIME_EOSGRID:time_increment = 30000 ;
%                TIME_EOSGRID:units = "minutes since 2015-07-21 00:00:00" ;
%                TIME_EOSGRID:long_name = "time" ;
%
%
%UM
%  REFTIME  => time(attribute='units')="hours since 2015-11-04T12:00:00Z"
%  STEPTIME => time(i) = int hours
%
%        int time(time) ;
%                time:long_name = "forecast time" ;
%                time:units = "hours since 2015-11-04T12:00:00Z" ;
%                time:GRIB_orgReferenceTime = "2015-11-04T12:00:00Z" ;
%                time:GRIB2_significanceOfRTName = "Start of forecast" ;
%                time:_CoordinateAxisType = "Time" ;
%
%STEPTIME=time double or int
%         frtime float
%REFTIME=time:units  "..YYYY.MM.DD.HH.MM.SS"
%        reftime

\clearpage
\section{Examples}
\subsection{MetSonde}
\subsection{MetTraj}
\subsection{ncMetcheck}

\clearpage

\section{Appendix}
\subsection{Met file codes}
\subsection{Variable codes}
\end{document}
